
\documentclass{article}[10pt]

\usepackage{amsmath}
\usepackage{amsfonts}
\usepackage{amssymb}
\usepackage{graphicx}
\usepackage{listings}

\def\codedefault {\ttfamily}

\lstset
{
  language=[ISO]C++,
  morekeywords={auto,constexpr,decltype,delete,interrupt,nullptr,noexcept,nothrow,static_assert},
  framerule=0.40pt,
  showstringspaces=false,
  extendedchars=true,
  basicstyle=\codedefault,
  commentstyle=\codedefault\itshape,
  keywordstyle=\codedefault\bfseries,
  frame=tb,
  aboveskip={1.1\baselineskip},
  belowskip={1.1\baselineskip}
}

\def\cppnineeight {C++$98$}
\def\cppothree    {C++$03$}
\def\cppeleven    {C++$11$}
\def\ceightnine   {C$89$}
\def\cninenine    {C$99$}

\def\trademarksymbolr   {$^{\text{\rm{\scriptsize{\textregistered}}}}$}
\def\trademarksymboltm  {$^{\text{\rm{\scriptsize{TM}}}}$}

\def\wolframalpha       {{WolframAlpha\trademarksymbolr}}

\begin{document}

\title{Zeros of Cylindrical Bessel and Neumann Functions}
\maketitle

\noindent
For every real order~$\nu$, cylindrical Bessel and Neumann
functions have an infinite number of zeros on the positive
real axis. The real zeros on the positive real axis
can be found by solving for the roots of

\begin{eqnarray}
J_{\nu}
\left(j_{\nu,\,m}\right)
&=&
0
\nonumber \\
Y_{\nu}
\left(y_{\nu,\,m}\right)
&=&
0
\,{\text{.}}
\end{eqnarray}

\noindent
Here, $j_{\nu,\,m}$ represents the~$m^{\text{th}}$
root of the cylindrical Bessel function of order~$\nu$
and $y_{\nu,\,m}$ represents the~$m^{\text{th}}$
root of the cylindrical Neumann function of order~$\nu$.

Various methods are used to compute initial estimates
for~$j_{\nu,\,m}$ and~$y_{\nu,\,m}$, and these will be described
in detail below. After finding the initial estimate of a given root,
its precision is subsequently refined to the desired level
using Newton-Raphson iteration from \lstinline|Boost.Math|'s
root-finding utilities combined with the functions
\lstinline|cyl_bessel_j()| and \lstinline|cyl_neumann()|,
also from \lstinline|Boost.Math|.

Newton iteration requires both $J_{\nu}(x)$ (or~$Y_{\nu}(x)$)
as well as its derivative. The derivatives of $J_{\nu}(x)$ and $Y_{\nu}(x)$
with respect to~$x$ are given by
Eq.~$9$.$1$.$3$ in~\cite{bibitem:abramowitz}.
In particular,

\begin{eqnarray}
\dfrac{d}{dx}J_{\nu}(x)
&=&
J_{\nu - 1}(x)
\,-\,
\frac{\nu}{x}\,J_{\nu}(x)
\nonumber \\
\dfrac{d}{dx}Y_{\nu}(x)
&=&
Y_{\nu - 1}(x)
\,-\,
\frac{\nu}{x}\,Y_{\nu}(x)
\,{\text{.}}
\end{eqnarray}

Enumeration of the rank of a root
(in other words the index of a root)
begins with one and counts up, in other words
$m\,=\,1,\,2,\,3,\ldots$,~etc. The value of the
first root is always greater than zero.

For certain special parameters, cylindrical Bessel functions
and cylindrical Neumann functions have
a root at the origin. For example,
$J_{\nu }(x)$ has a root at the origin for every positive order
$\nu\,>\,0$ and for every negative integer order
$\nu\,=\,-n$, with $n\,\in\,\mathbb{N}^{+}$
and $n\,\ne\,0$.
In addition, $Y_{\nu }(x)$ has a root at the origin
for every negative half-integer order
$\nu\,=\,-n/\,2$, with $n\,\in\,\mathbb{N}^{+}$
and $n\,\ne\,0$.
For these special parameter values, the origin with
a value of~$x\,=\,0$ is provided as the $0^{\text{th}}$
root generated by {\lstinline|cyl_bessel_j_zero()|}
and {\lstinline|cyl_neumann_zero()|}.

When calculating initial estimates for the roots
of Bessel functions, a distinction is made between
positive order and negative order, and different
methods are used for these. In addition, different algorithms
are used for the first root ($m\,=\,1$) and
for subsequent roots with higher rank ($m\,\ge\,2$).
Furthermore, estimates of the roots for Bessel functions
with order above and below a cutoff at~$\nu\,=\,2.2$
are calculated with different methods.

Calculations of the estimates of $j_{\nu,\,1}$ and $y_{\nu,\,1}$
with~$0\,\le\,\nu\,<\,2.2$ use empirically tabulated values.
The coefficients for these have been generated by a
computer algebra system.

Calculations of the estimates of $j_{\nu,\,1}$ and $y_{\nu,\,1}$
with~$\nu\,\ge\,2.2$ use
Eqs.~$9$.$5$.$14$ and~$9$.$5$.$15$
in~\cite{bibitem:abramowitz}.
In particular,

\begin{eqnarray}
j_{\nu,\,1}
& \sim &
\nu
\,+\,
1.85575\,71\nu^{\frac{1}{3}}
\,+\,
1.03315\,0\nu^{-\frac{1}{3}}
\,-\,
0.00397\nu^{-1}
\nonumber \\
&&
\,-\,
0.0908\nu^{-\frac{5}{3}}
\,+\,
0.043\nu^{-\frac{7}{3}}
\,+\,\cdots
\,{\text{,}}
\end{eqnarray}

\noindent
and

\begin{eqnarray}
y_{\nu,\,1}
& \sim &
\nu
\,+\,
0.93157\,68\nu^{\frac{1}{3}}
\,+\,
0.26035\,1\nu^{-\frac{1}{3}}
\,+\,
0.01198\nu^{-1}
\nonumber \\
&&
\,-\,
0.0060\nu^{-\frac{5}{3}}
\,-\,
0.001\nu^{-\frac{7}{3}}
\,+\,\cdots
\,{\text{,}}
\end{eqnarray}

Calculations of the estimates of $j_{\nu,\,m}$ and $y_{\nu,\,m}$
with rank $m\,\ge\,2$ and~$0\,\le\,\nu\,<\,2.2$ use
McMahon's approximation, as described in Sect.~$9$.$5$
of~\cite{bibitem:abramowitz} and Eq.~$9$.$5$.$12$ therein.
In particular,

\begin{eqnarray}
j_{\nu,\,m},\,y_{\nu,\,m}
& \sim &
\beta
\,-\,
\frac{\mu - 1}{8\beta}
\,-\,
\frac{4(\mu - 1)(7\mu -31)}{3(8\beta)^3}
\nonumber \\
&&
\,-\,
\frac{32(\mu - 1)(83\mu^2 - 982\mu +3779)}{15(8\beta)^5}
\nonumber \\
&&
\!\!\!\! - \,
\frac{64(\mu - 1)(6949\mu^3 - 1\,53855\mu^2 + 15\,85743\mu - 62\,77237)}{105(8\beta)^7}
\nonumber \\
&&
\,-\,\cdots
\,{\text{,}}
\label{equation:mcmahon}
\end{eqnarray}

\noindent
where $\mu\,=\,4\nu^2$
and $\beta\,=\,\left(m\,+\,\frac{1}{2}\nu\,-\,\frac{1}{4}\right)\pi$
for $j_{\nu,\,m}$
and $\beta\,=\,\left(m\,+\,\frac{1}{2}\nu\,-\,\frac{3}{4}\right)\pi$
for $y_{\nu,\,m}$.

\vspace{2pt}

Calculations of the estimates of $j_{\nu,\,m}$ and $y_{\nu,\,m}$
with~$\nu\,\ge\,2.2$ use
one term in the asymptotic expansion given in
Eq.~$9$.$5$.$22$ and the upper line of Eq.~$9$.$5$.$26$
combined with Eq.~$9$.$3$.$39$, all in~\cite{bibitem:abramowitz}.
The latter two equations are expressed for
argument~$x$ greater than one. In summary,

\begin{equation}
j_{\nu,\,m}
\,\sim\,
\nu x(-\zeta)
\,+\,
\dfrac{f_{1}(-\zeta)}{\nu}
\,{\text{,}}
\end{equation}

\noindent
where $-\zeta\,=\,\nu^{-2/\,3}a_{m}$ and $a_{m}$ is
the absolute value of the $m^{\text{th}}$ root
of $Ai(x)$ on the negative real axis.
Here, $x\,=\,x(-\zeta)$ is the inverse of the function

\begin{equation}
\dfrac{2}{3}
(-\zeta)^{3/\,2}
\,=\,
\sqrt{x^{2}\,-\,1}
\,-\,
\cos^{-1}\left(\dfrac{1}{x}\right)
\,{\text{.}}
\label{equation:uniform}
\end{equation}

\noindent
Furthermore,

\begin{equation}
f_1(-\zeta)
\,=\,
\dfrac{1}{2}\,
x(-\zeta)
\bigl\{h(-\zeta)\bigr\}^{2}
b_{0}(-\zeta)\,
{\text{,}}
\end{equation}

\noindent
where

\begin{equation}
h(-\zeta)
\,=\,
\left\{
\dfrac{4(-\zeta)}{x^2\,-\,1}
\right\}^{\frac{1}{4}}\,
{\text{,}}
\end{equation}

\noindent
and

\begin{equation}
b_{0}(-\zeta)
\,=\,
-\dfrac{5}{48\zeta^{2}}
\,+\,
\dfrac{1}{(-\zeta)^{\frac{1}{2}}}
\Biggl\{
\dfrac{5}{24(x^{2}\,-\,1)^{3/\,2}}
\,+\,
\dfrac{1}{8(x^{2}\,-\,1)^{1/\,2}}
\Biggr\}
{\text{.}}
\end{equation}

When solving for $x(-\zeta)$ in Eq.~\ref{equation:uniform} above,
the right-hand-side is expanded to $O(x^{2})$ in
a Taylor series for large~$x$. This results in

\begin{equation}
\dfrac{2}{3}
(-\zeta)^{3/\,2}
\,\approx\,
x
\,+\,
\dfrac{1}{2x}
\,-\,
\dfrac{\pi}{2}\,
{\text{.}}
\end{equation}

\noindent
The positive root of the resulting quadratic equation
is used to find an initial estimate~$x(-\zeta)$.
This initial estimate is subsequently refined with
several steps of Newton-Raphson iteration
in Eq.~\ref{equation:uniform}.

Estimates of the roots of cylindrical Bessel functions
of negative order on the positive real axis are found
using interlacing relations. For example, the~$m^{\text{th}}$
root of the cylindrical Bessel function $j_{-\nu,m}$
is bracketed by the $m^{\text{th}}$ root and the
$(m\,+\,1)^{\text{th}}$ root of the Bessel function of
corresponding positive integer order. In other words,

\begin{equation}
j_{n_{\nu},m}
\,<\,
j_{-\nu,m}
\,<\,
j_{n_{\nu},m\,+\,1}
\end{equation}

\noindent
where $m\,>\,1$ and $n_{\nu}$ represents the integral
floor of the absolute value of~$\left|-\nu\right|$.

Similar bracketing relations are used to find estimates
of the roots of Neumann functions of negative order,
whereby a discontinuity at every negative half-integer
order needs to be handled.

Bracketing relations do not hold for the first root
of cylindrical Bessel functions and cylindrical Neumann
functions with negative order. Therefore, iterative algorithms
combined with root-finding via bisection are used
to localize $j_{-\nu,1}$ and $y_{-\nu,1}$.

\begin{thebibliography}{9}

\bibitem{bibitem:abramowitz}
M.~Abramowitz and I.~A.~Stegun:
{\textit{Handbook of Mathematical Functions, $9^{th}$ Printing}},
(Dover Publications, New York, 1972)

\end{thebibliography}

\end{document}
