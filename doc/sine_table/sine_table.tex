\documentclass{article}[10pt]

\usepackage{amsmath}
\usepackage{amsfonts}
\usepackage{amssymb}
\usepackage{graphicx}
\usepackage{listings}

\def\codedefault {\ttfamily}

\lstset
{
  language=[ISO]C++,
  morekeywords={interrupt,static_assert,constexpr,delete,auto,decltype,nullptr,noexcept},
  framerule=0.40pt,
  showstringspaces=false,
  extendedchars=true,
  basicstyle=\codedefault,
  commentstyle=\codedefault\itshape,
  keywordstyle=\codedefault\bfseries,
  frame=tb,
  aboveskip={1.1\baselineskip},
  belowskip={1.1\baselineskip}
}

\def\cppnineeight {C++$98$}
\def\cppothree    {C++$03$}
\def\cppox        {C++$11$}
\def\cninenine    {C$99$}

\def\trademarksymbolr   {$^{\text{\rm{\scriptsize{\textregistered}}}}$}
\def\trademarksymboltm  {$^{\text{\rm{\scriptsize{TM}}}}$}

\def\mathematica        {{Mathematica\trademarksymbolr}}
\def\wolframalpha       {{WolframAlpha\trademarksymbolr}}

\newcommand{\deriveval}[2][]{#1|_{#2}}

\begin{document}

\title{Making a Sine Table with {\codedefault{Boost.Multiprecision}}}
\maketitle

\noindent
We will now describe a variety of ways to use
\lstinline|Boost.Math| in combination with
\lstinline|Boost.Multiprecision|
in order to perform floating-point numerical calculations
with high precision.

The \lstinline|Boost.Multiprecision| library can be used
for computations requiring precision exceeding that
of standard built-in types such as
\lstinline|float|, \lstinline|double| and \lstinline|long double|.
For extended-precision calculations,
\lstinline|Boost.Multiprecision| supplies a template data
type called \lstinline|cpp_dec_float|.
The number of decimal digits of precision is fixed at compile-time
via template parameter. For example,

\begin{lstlisting}
#include <boost/multiprecision/cpp_dec_float.hpp>

using boost::multiprecision::cpp_dec_float;

// A floating-point type with 50 digits of precision.
typedef cpp_dec_float<50> float50;
\end{lstlisting}

Here, we have defined the local data type
\lstinline|float50| with with~$50$ decimal digits of
precision. We can use this data type to print, for example,
the value of $1/\,7$ to~$50$ digits.

\begin{lstlisting}
#include <iostream>
#include <limits>

// 1/7 with 50 digits of precision.
float50 seventh = float50(1) / 7;

const int d
  = std::numeric_limits<float50>::digits10;

// Print 1/7 with 50 digits of precision.
std::cout << std::setprecision(d)
          << seventh
          << std::endl;
\end{lstlisting}

For the sake of convenience, \lstinline|Boost.Multiprecision|
defines a variety of data types with fixed precision.
These include, among others, data types with~$50$ and~$100$
decimal digits of precision, respectively. In particular,

\begin{lstlisting}
using boost::multiprecision;

cpp_dec_float_50  f50;  // Has  50 digits.
cpp_dec_float_100 f100; // Has 100 digits.
\end{lstlisting}

For more information on the \lstinline|cpp_dec_float| data type,
see the tutorial in \lstinline|Boost.Multiprecision|.

Mathematical software usually requires one or more
known constant values such as~$\pi$ or $\log{2}$,~etc.
It often makes sense
to initialize a constant value stored in a built-in type
from a multiprecision value of higher precision.
This guarantees that the built-in type will be initialized
to the the very last bit of its own precision.

We will now obtain the value of~$\pi$ with~$50$ decimal
digits of precision from \lstinline|Boost.Math|'s
template \lstinline|pi()| function.

\begin{lstlisting}
using boost::multiprecision;

cpp_dec_float_50 pi_50;
  = boost::math::constants::pi<cpp_dec_float_50>();
\end{lstlisting}

A multiprecision value can be converted to a built-in type
using the member function \lstinline|convert_to()|.
In particular, the code below initializes a
value of type \lstinline|long double| from
a multiprecision value having~$50$ digits of precision.

\begin{lstlisting}
const long double pi_ld
  = pi_50.convert_to<long_double>();
\end{lstlisting}

One usually needs to compute tables of numbers in
mathematical software. A fast Fourier transform (FFT),
for example, may use a table of the values of
$\sin\left({\pi /\,2^n}\right)$
in its implementation details. In order to maximize the
precision in the FFT implementation, the precision of the
tabulated trigonometric values should exceed that of the
built-in floating-point type used in the FFT.

The sample below computes a table of the values of
$\sin\left({\pi /\,2^n}\right)$ in the range
$1\,\le\,n\le\,31$.
A precision of~$50$ decimal digits is used.

\begin{lstlisting}
#include <vector>
#include <algorithm>
#include <iostream>
#include <iomanip>
#include <iterator>
#include <limits>
#include <boost/multiprecision/cpp_dec_float.hpp>

using boost::multiprecision::cpp_dec_float_50;
using boost::math::constants::pi;

typedef cpp_dec_float_50 float50;

int main(int, char**)
{
  std::vector<float50> sin_values(31U);

  unsigned n = 1U;

  // Generate the sine values.
  std::for_each(
    sin_values.begin(),
    sin_values.end(),
    [&n](float50& y)
    {
      y = sin(pi<float50>() / pow(float50(2), n));
      ++n;
    });

  // Set the output precision.
  const int d
    = std::numeric_limits<float50>::digits10;

  std::cout.precision(d);

  // Print the sine table.
  std::copy(sin_values.begin(),
            sin_values.end(),
    std::ostream_iterator<float50>(std::cout, "\n"));
}
\end{lstlisting}

This program makes use of, among other program elements,
the data type \lstinline|cpp_dec_float_50| from
\lstinline|Boost.Multiprecision|, the value of $\pi$
retrieved from \lstinline|Boost.Math|
and a \cppox\ lambda function combined with
\lstinline|std::for_each()|.

\pagebreak

The output of this program is shown below.

\begin{lstlisting}
1
0.70710678118654752440084436210484903928483593768847
0.38268343236508977172845998403039886676134456248563
0.19509032201612826784828486847702224092769161775195
0.098017140329560601994195563888641845861136673167501
0.049067674327418014254954976942682658314745363025753
0.024541228522912288031734529459282925065466119239451
0.012271538285719926079408261951003212140372319591769
0.0061358846491544753596402345903725809170578863173913
0.003067956762965976270145365490919842518944610213452
0.0015339801862847656123036971502640790799548645752374
0.00076699031874270452693856835794857664314091945206328
0.00038349518757139558907246168118138126339502603496474
0.00019174759731070330743990956198900093346887403385916
9.5873799095977345870517210976476351187065612851145e-05
4.7936899603066884549003990494658872746866687685767e-05
2.3968449808418218729186577165021820094761474895673e-05
1.1984224905069706421521561596988984804731977538387e-05
5.9921124526424278428797118088908617299871778780951e-06
2.9960562263346607504548128083570598118251878683408e-06
1.4980281131690112288542788461553611206917585861527e-06
7.4901405658471572113049856673065563715595930217207e-07
3.7450702829238412390316917908463317739740476297248e-07
1.8725351414619534486882457659356361712045272098287e-07
9.3626757073098082799067286680885620193236507169473e-08
4.681337853654909269511551813854009695950362701667e-08
2.3406689268274552759505493419034844037886207223779e-08
1.1703344634137277181246213503238103798093456639976e-08
5.8516723170686386908097901008341396943900085051757e-09
2.9258361585343193579282304690689559020175857150074e-09
1.4629180792671596805295321618659637103742615227834e-09
\end{lstlisting}

This output can be copied
as text and readily integrated into a given source code file.
Alternatively, the output can be written to a text file or
even be used as part of a self-written automatic code generator.

A computer algebra system can be used to verify
the results obtained from
\lstinline|Boost.Math| and \lstinline|Boost.Multiprecision|.
For example, the \mathematica\
computer algebra system~\cite{bibitem:mathematica}
can obtain a similar table with the command:

\begin{verbatim}
Table[N[Sin[Pi / (2^n)], 50], {n, 1, 31, 1}]
\end{verbatim}

The \wolframalpha\
Computational Knowledge Engine~\cite{bibitem:wolframalpha}
can also be used to generate this table.
The same command can be used.

In general, calculations of special functions inherently
lose a few bits of precision (or more)
due to cancellations in series expansions,
recurrence relations and the like. It often arises, however,
that data tables originating from
special function calculations are used as the basis
of an even higher-level calculation. Computing data tables
with extended precision can improve the overall accuracy of
these kinds of calculations because the data have
more precision than necessary.

It may be wise to warm-cache pre-computed high-precision
table values in a static container such as \lstinline|std::vector|.
It can also be convenient to store pre-computed high-precision
data tables in text-based files that can be parsed whenever
the values need to be initialized and warm-cached.

\begin{thebibliography}{9}
\bibitem{bibitem:mathematica}
Wolfram: {\textit{Wolfram \mathematica}},\\
{\ttfamily{http://www.wolfram.com/mathematica}} (2013)
\bibitem{bibitem:wolframalpha}
Wolfram: {\textit{\wolframalpha\ Comutational Knowledge Engine}},\\
{\ttfamily{http://www.wolframalpha.com}} (2013)
\end{thebibliography}

\end{document}
